\documentclass[12pt]{amsart}
\usepackage{geometry}
\geometry{margin=1in}

\usepackage{hyperref}

\title{Mathematics in the time of MOOCs: Post Workshop Report}

\author[Bonfert-Taylor]{Petra Bonfert-Taylor (Wesleyan)}
\author[Chadam]{John Chadam (Pittsburgh)}
\author[Fowler]{Jim Fowler (Ohio State)}

\usepackage[style=numeric]{biblatex}
\addbibresource{references.bib}

\usepackage{multicol}

\begin{document}
\maketitle

\section{Summary}

At present there is a great deal of optimism, as well as concern,
within the mathematics community about the growing presence of MOOCs.
As a result, there are many MOOC related conferences focusing on the
broader policy issues or on specific approaches, but heretofore
missing is a venue for those \textit{creating} MOOCs in mathematics to
come together to plan future courses, to integrate course content, to
share course data, to share technical and conceptual ideas, and to
review content, with an overarching goal to accelerate and optimize
MOOC content creation.

This proposal is an attempt to remedy this by providing such a venue
for MOOC content creators in the Mathematical Sciences to discuss
building MOOCs for mathematics, and to discuss establishing an ``AIM
Open MOOC Initiative'' which would review MOOC content, as well as
make MOOC development tools, course content, course data, and
assessment instruments available for reuse.

\section{Need for the workshop}

Research universities are ``doing the worst job in maintaining student
confidence in their mathematical abilities, enjoyment of mathematics,
and interest in continuing with the mathematics that is needed to
pursue their intended careers'' \parencite{calculus-students}.
MOOCs---already being used at some of the most prestigious US research
universities \parencite{morris2013moocs}---are generating high-level
interest because they are believed to provide a method by which
universities can rein in costs and improve student
outcomes \parencite{bowen2013higher}.  That high-level of interest
among stakeholders is evidenced by the
\begin{itemize}
\item many colloquium talks at mathematics departments,
\item panel discussions at professional society meetings, such as
  \textit{Two Worlds Collide: MOOCs and the Ivory Tower} at the 2014
  AMS--MAA Joint Meeting, or the \textit{AMS Conference on MOOCs} in
  Washington, D.C., October 22, 2013,
\item panel discussions at granting agencies, such as the
  December~2012 NSF Internal Workshop on MOOCs at
  \url{http://www.nsf.gov/attachments/127404/public/Engineering_Education.pdf},
\item white papers generated from such panel discussions, for example,
  INGenIOuS at \url{http://ingeniousmathstat.org/technology-and-moocs}, and
\item engineering and education sponsored workshops, such as the \textit{National Academy
  of Engineering Conference} in Irvine, California, October 29, 2013, or the \textit{MOOC
  Conference} at UT--Arlington on December 4--5, 2013.
\end{itemize}
Note that the structure and venues of these activities are chosen
purposely to lead to discussions of a wide range of topics from
explaining the implementation of a particular MOOC to potential users
and students, to discussing the constraints that might emerge for
non-mathematical reasons from educators, publishers, platform
providers (Coursera, edX, Udacity), university administrators and
politicians.

The aforementioned events are primarily policy or pedagogy
conversations; the missing ingredient has been mathematicians.  That
is not to say that mathematicians aren't heavily involved, but simply
that the focus on previous MOOC related events has not been the
mathematics, but the surrounding policy or pedagogical issues.  The
quiescence of mathematicians is not a new phenomenon, and is not
restricted to MOOCs.  For example, the President's Council of Advisors
on Science and Technology issued a report calling for one million
additional STEM graduates \cite{engage-to-excel}, which the National
Academy of Sciences responded to with a discussion of discipline-based
education research \cite{dber-report}, which, while addressing
biology, chemistry, physics, and geosciences, notably did not address
mathematics qua mathematics, but only mathematics as yet another tool.
If mathematics is being presented as merely a tool, rather than
something worth studying on its own merits, rather than as a highlight
of the human experience, then is it any surprise that mathematics is
unpopular with students?  Students are abandoning STEM majors because
of the low quality of instruction ``with calculus often cited as a
primary reason'' \parencite{calculus-programs}.  Since MOOCs are one
significant means towards improving instructional quality, it behooves
the mathematical community to take charge of the exposition in our
MOOCs---before someone outside of mathematics does our job for us.

\section{Objectives}

This is a proposal for a series of events exploring the opportunities
and challenges of teaching Mathematics at the university level using
MOOCs. The objective is to bring together academics who have created
or are creating MOOCs in the Mathematical Sciences to share their
expertise and explore how best to build MOOCs.  The overarching goal
is to share resources between MOOC content creators, which means
sharing \textbf{content, courses, tools, platforms, and data.}

\subsection{Sharing content.}  AIM's Open Textbook Initiative is a
great and already successful example of how institutions can share
open resources \cite{aim-notices}.  The proposal would extend AIM's
textbook initiative into an ``Open MOOC Initiative.''  This turn of
phrase may initially appear redundant: doesn't one of the Os in
``MOOC'' stand for open?  MOOCs are ``open'' in that enrollment is
free, which means MOOCs are usually ``gratis'' but not necessarily
``libre'' \cite{suber2008gratis}.  But as AIM's Open Textbook
Initiative illustrates, ``libre'' can be crucial.  Whether MOOCs ought
to be ``open'' in that more expansive sense of ``libre'' is a
significant issue worth debating.

There are other reasons to consider an expansion of textbook
initiative.  AIM's Open Textbook Initiative is no longer providing
direct substitutes to the textbook as it is currently peddled, because
textbook publishers are not providing just a book anymore; so-called
``textbooks'' are bundled with packages of lecture notes, videos,
online, often adaptive, homework system, interactive JavaScript demos,
etc.  Insofar as MOOCs are seen not as courses but as repositories of
content, MOOCs are closer to what the publishers are actually offering
as textbooks.  So the open textbook movement could respond by offering
superior, critically reviewed, fully open resources under the moniker
of MOOCs.

An Open MOOC Initiative may begin by primarily acting as a reviewer of
MOOC content---whether or not it is reusable content---but, based on
the discussions at the workshop, it may grow into a central repository
of open, reusable content for MOOCs and other blended forms of
instruction.  A major objective of the proposed workshop is to
consider how best to evaluate and review MOOCs, and then how the
interested parties could organize and share their reusable content in
a central repository.

Some of the challenges are technical, which is one reason why the
participant list includes not just mathematicians, but also
technologists.  For example, there are significant technical
challenges with video, so it will be useful for MOOC content creators
to discuss their workflow for handling complicated content formats
like video, even if the content creators are less interested in making
their videos reusable.  Sometimes the mathematicians are divorced from
such issues because their institutions are using instructional
designers, but it is nonetheless important for the mathematicians to
be aware of the issues.

\subsection{Sharing courses.}  Not everyone will be interested in
sharing course content.  Nevertheless, multi-institution and
multi-platform courses could pool the otherwise limited resources of
isolated departments and MOOC content authors.  Currently, it is
challenging for an interested MOOC author to determine what courses
other institutions may be offering or may be planning to offer;
discussing what upper-division courses make the most sense for
MOOCification is a major objective of the workshop.  It is important
that such discussions about courses be made between mathematicians,
instead of simply by considering enrollment numbers or potential
revenue.  How can a proof-based course be run in a MOOC format?  Are
there missed opportunities for making connections between MOOCs at
different institutions?

Finally, many interested MOOC authors may find themselves to be the
only MOOC author at their institution.  For such authors,
multi-institution coordination may be the only way to get into the
game.  The promise of the internet is not only to bring students
together, but to make it possible for educators at distant locations
to teach together.  Thus, a major objective of the workshop is to help
connect authors who would be interested in collaborating on a
multi-institutional course.  Many significant MOOCs have already grown
out of collaborations.

\subsection{Sharing tools.} Even for MOOC content creators who are not
interested in sharing content or collaborating on a course, tools can
be shared.  There are already many tools that could be shared between
institutions, but perhaps these tools are not well known.  For
example, Jim Fowler built the ``autocutter'' available at
\url{https://github.com/kisonecat/autocut} which automates the process
of video editing for many MOOCs by applying a fast Fourier transform
to the audio, looking for human voices, and cutting away the parts of
the initial and final segments of video that do not include a human
voice.  There are other useful tools for superimposing graphics onto
video.  What tools are other MOOC content creators using?  Are they
building tools that can be reused by others?  If so, these tools could
be used very broadly.

The workshop includes not only mathematicians, but also technologists,
precisely because mathematicians may not be in the best position to
understand some of the technical challenges surrouding MOOCs.  An
objective for the workshop is to establish collaborations not only
between MOOC content creators, but also between technologists and
mathematicians, with the aim of creating more engaging methods for
students to do mathematics online.

\subsection{Sharing platforms.}  For MOOCs to succeed, there must be
``customizable, sustainable platforms'' \parencite{bowen2013higher}
for online instruction.  To what extent are the current platforms
customizable?  Do MOOC content creators believe that they are creating
``sustainable'' content which will be available---and reusable---for
decades to come?  \TeX\ is perhaps the best example of a truly
sustainable platform, and consequently, mathematicians are comfortable
putting considerable effort into the creation of \TeX\ documents.  How
can content creators create content for the web that will last the
test of time?  Many of these questions address the anxiety that
prevents our colleagues from investing their time in creating MOOC
content.

There are also questions about how best to use various MOOC platforms.
For example, many Coursera courses make significant use of Community
TAs, but the effectiveness of such TAs varies.  A key objective of the
workshop is to share best practices not just for creating content, but
practically running and organizing courses centered around that
content.  How should we use the Community TAs?  How can we make it
easier to ``rerun'' old courses?  How can MOOC content creators
transport our content from one platform to another?

There are also opportunities for building new platforms.  What would
an ideal MOOC platform look like, for a course in mathematics?
Certainly, a platform for mathematics can be re-used for other
courses, just as pieces of MOOCulus were reused for the English
WexMOOC \cite{gates-foundation-grant}.  Nevertheless, there are
math-specific desiderata for a MOOC platform that may not be currently
available on any of the major platforms, so a major objective of the
workshop is to formalize such features and bring them before the
platform developers.

\subsection{Sharing data.} What student data will be mined?  What
sorts of data will be stored on each student interaction?  And how can
such data be shared between institutions?  How are MOOC content
creators improving their courses by relying on such data?  Such
decisions, if made poorly, may seriously constrain the usefulness of
MOOCs.  Some have chosen to record the date and time the student has
engaged with various educational
applications \parencite{RomeroZaldivar20121058}; others are looking at
clickstream data \parencite{boyer2013student}.  Going forward, it is
worth considering a format for heterogeneous data logging for student
interaction, like the Tin Cap API \parencite{tin-can-api}.  There may
be advantages to gathering data from other sources, such as video
views from within Coursera's platform or forum posts, but there can be
challenges to getting such data from the platforms as they currently
exist.

Mathematicians, in contrast to many other MOOC authors, are in a
strong position to both generate the course content and then, armed
with domain knowledge, analyze the student data that will lead to
refinements in the student outcomes for their courses.  A major
objective of the workshop is to share best practices for doing so.
Moreover, it is likely that an Open MOOC Initiative, beyond
coordinating the reuse of course content, could also become a
clearinghouse for student data from MOOCs.

%%%%%%%%%%%%%%%%%%%%%%%%%%%%%%%%%%%%%%%%%%%%%%%%%%%%%%%%%%%%%%%%
\section{Plan}

To achieve these objectives, the proposal is to have a core group of
approximately 25~individuals from the list in Appendix~A participate
in an initial 2--3 day organizational workshop to be held at AIM in
Palo Alto.

The meeting at AIM would be scheduled for a few days between February
2--9.

\subsection{Initial AIM workshop}

Overall, the discussion topics will be the current state of MOOCs in
mathematics, leading to an exploration of what needs to be done and
how these objectives could be accomplished.  To frame the activities,
the workshop begins with an introductory hour, in which participants
give a very brief ($<2$ minute) overview of what MOOCs they have run
or are planning to run.  The majority of the participants are
mathematicians who have run a MOOC on a variety of platforms, but
there are also participants with a strong technical background, with a
background in open educational resources, or with corporate
experience.

Smaller discussion groups of between 4 and 6 participants would then
form on the first day around the major themes of content, courses,
tools, platforms, and data.  These smaller groups would report to the
larger group, with a goal of producing a short report (collaboratively
edited using the Sagemath Cloud) giving practical next steps for
sharing content, building collaborations for multi-institution
courses, disseminating tools for MOOC creation, using the various MOOC
platforms, and sharing data.

The final day of the workshop focuses on the dissemination of the
results by involving the professional societies, and working through
the many issues and opinions that will necessarily surround any form
an Open MOOC Initiative could take.

\newpage
\printbibliography

\end{document}

%%% Local Variables: 
%%% mode: latex
%%% TeX-master: t
%%% End: 
